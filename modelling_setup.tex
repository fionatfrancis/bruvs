% Options for packages loaded elsewhere
\PassOptionsToPackage{unicode}{hyperref}
\PassOptionsToPackage{hyphens}{url}
%
\documentclass[
]{article}
\usepackage{amsmath,amssymb}
\usepackage{lmodern}
\usepackage{iftex}
\ifPDFTeX
  \usepackage[T1]{fontenc}
  \usepackage[utf8]{inputenc}
  \usepackage{textcomp} % provide euro and other symbols
\else % if luatex or xetex
  \usepackage{unicode-math}
  \defaultfontfeatures{Scale=MatchLowercase}
  \defaultfontfeatures[\rmfamily]{Ligatures=TeX,Scale=1}
\fi
% Use upquote if available, for straight quotes in verbatim environments
\IfFileExists{upquote.sty}{\usepackage{upquote}}{}
\IfFileExists{microtype.sty}{% use microtype if available
  \usepackage[]{microtype}
  \UseMicrotypeSet[protrusion]{basicmath} % disable protrusion for tt fonts
}{}
\makeatletter
\@ifundefined{KOMAClassName}{% if non-KOMA class
  \IfFileExists{parskip.sty}{%
    \usepackage{parskip}
  }{% else
    \setlength{\parindent}{0pt}
    \setlength{\parskip}{6pt plus 2pt minus 1pt}}
}{% if KOMA class
  \KOMAoptions{parskip=half}}
\makeatother
\usepackage{xcolor}
\usepackage[margin=1in]{geometry}
\usepackage{color}
\usepackage{fancyvrb}
\newcommand{\VerbBar}{|}
\newcommand{\VERB}{\Verb[commandchars=\\\{\}]}
\DefineVerbatimEnvironment{Highlighting}{Verbatim}{commandchars=\\\{\}}
% Add ',fontsize=\small' for more characters per line
\usepackage{framed}
\definecolor{shadecolor}{RGB}{248,248,248}
\newenvironment{Shaded}{\begin{snugshade}}{\end{snugshade}}
\newcommand{\AlertTok}[1]{\textcolor[rgb]{0.94,0.16,0.16}{#1}}
\newcommand{\AnnotationTok}[1]{\textcolor[rgb]{0.56,0.35,0.01}{\textbf{\textit{#1}}}}
\newcommand{\AttributeTok}[1]{\textcolor[rgb]{0.77,0.63,0.00}{#1}}
\newcommand{\BaseNTok}[1]{\textcolor[rgb]{0.00,0.00,0.81}{#1}}
\newcommand{\BuiltInTok}[1]{#1}
\newcommand{\CharTok}[1]{\textcolor[rgb]{0.31,0.60,0.02}{#1}}
\newcommand{\CommentTok}[1]{\textcolor[rgb]{0.56,0.35,0.01}{\textit{#1}}}
\newcommand{\CommentVarTok}[1]{\textcolor[rgb]{0.56,0.35,0.01}{\textbf{\textit{#1}}}}
\newcommand{\ConstantTok}[1]{\textcolor[rgb]{0.00,0.00,0.00}{#1}}
\newcommand{\ControlFlowTok}[1]{\textcolor[rgb]{0.13,0.29,0.53}{\textbf{#1}}}
\newcommand{\DataTypeTok}[1]{\textcolor[rgb]{0.13,0.29,0.53}{#1}}
\newcommand{\DecValTok}[1]{\textcolor[rgb]{0.00,0.00,0.81}{#1}}
\newcommand{\DocumentationTok}[1]{\textcolor[rgb]{0.56,0.35,0.01}{\textbf{\textit{#1}}}}
\newcommand{\ErrorTok}[1]{\textcolor[rgb]{0.64,0.00,0.00}{\textbf{#1}}}
\newcommand{\ExtensionTok}[1]{#1}
\newcommand{\FloatTok}[1]{\textcolor[rgb]{0.00,0.00,0.81}{#1}}
\newcommand{\FunctionTok}[1]{\textcolor[rgb]{0.00,0.00,0.00}{#1}}
\newcommand{\ImportTok}[1]{#1}
\newcommand{\InformationTok}[1]{\textcolor[rgb]{0.56,0.35,0.01}{\textbf{\textit{#1}}}}
\newcommand{\KeywordTok}[1]{\textcolor[rgb]{0.13,0.29,0.53}{\textbf{#1}}}
\newcommand{\NormalTok}[1]{#1}
\newcommand{\OperatorTok}[1]{\textcolor[rgb]{0.81,0.36,0.00}{\textbf{#1}}}
\newcommand{\OtherTok}[1]{\textcolor[rgb]{0.56,0.35,0.01}{#1}}
\newcommand{\PreprocessorTok}[1]{\textcolor[rgb]{0.56,0.35,0.01}{\textit{#1}}}
\newcommand{\RegionMarkerTok}[1]{#1}
\newcommand{\SpecialCharTok}[1]{\textcolor[rgb]{0.00,0.00,0.00}{#1}}
\newcommand{\SpecialStringTok}[1]{\textcolor[rgb]{0.31,0.60,0.02}{#1}}
\newcommand{\StringTok}[1]{\textcolor[rgb]{0.31,0.60,0.02}{#1}}
\newcommand{\VariableTok}[1]{\textcolor[rgb]{0.00,0.00,0.00}{#1}}
\newcommand{\VerbatimStringTok}[1]{\textcolor[rgb]{0.31,0.60,0.02}{#1}}
\newcommand{\WarningTok}[1]{\textcolor[rgb]{0.56,0.35,0.01}{\textbf{\textit{#1}}}}
\usepackage{graphicx}
\makeatletter
\def\maxwidth{\ifdim\Gin@nat@width>\linewidth\linewidth\else\Gin@nat@width\fi}
\def\maxheight{\ifdim\Gin@nat@height>\textheight\textheight\else\Gin@nat@height\fi}
\makeatother
% Scale images if necessary, so that they will not overflow the page
% margins by default, and it is still possible to overwrite the defaults
% using explicit options in \includegraphics[width, height, ...]{}
\setkeys{Gin}{width=\maxwidth,height=\maxheight,keepaspectratio}
% Set default figure placement to htbp
\makeatletter
\def\fps@figure{htbp}
\makeatother
\setlength{\emergencystretch}{3em} % prevent overfull lines
\providecommand{\tightlist}{%
  \setlength{\itemsep}{0pt}\setlength{\parskip}{0pt}}
\setcounter{secnumdepth}{-\maxdimen} % remove section numbering
\ifLuaTeX
  \usepackage{selnolig}  % disable illegal ligatures
\fi
\IfFileExists{bookmark.sty}{\usepackage{bookmark}}{\usepackage{hyperref}}
\IfFileExists{xurl.sty}{\usepackage{xurl}}{} % add URL line breaks if available
\urlstyle{same} % disable monospaced font for URLs
\hypersetup{
  pdftitle={FISH ABUNDANCE AND RICHNESS MODELS},
  hidelinks,
  pdfcreator={LaTeX via pandoc}}

\title{FISH ABUNDANCE AND RICHNESS MODELS}
\author{}
\date{\vspace{-2.5em}2023-03-06}

\begin{document}
\maketitle

This is a file to set up the BRUVs data to run linear mixed effects
models on fish abundance and lengths and to look at how these aspects of
fish communities differ between sponge and rock reef habitats and also
determine what are the main drivers that lead to these differences.

\hypertarget{fish-abundance}{%
\section{Fish abundance}\label{fish-abundance}}

Let's first look at fish abundance. This was measured using MaxN which
is a common metric used in baited video studies and represented the
maximum number of fish of a given species seen at one time in the frame
over a 60 min time period. I would like to look at total number of
non-schooling rockfish combined and then if there is enough data we
could look at an individual species (like quillback rockfish). We could
also look at the whole fish community (so including greenlings and
gobies and yellowtail rockfish etc) if we think that this is well
represented by BRUVs as a method?

\hypertarget{research-questions}{%
\subsection{Research Questions}\label{research-questions}}

\begin{enumerate}
\def\labelenumi{\arabic{enumi}.}
\item
  Does rockfish abundance differ between rocky reefs, sponge gardens and
  sponge reefs? (*Do we have enough data on sponge reefs? Should we add
  them to bioherms? rocky reefs? Hmmm question for discussion)
\item
  What are the main drivers of rockfish abundance (this question can be
  answered regardless of if there was a difference in habitat type or
  not)? The metrics we have include max \% cover of refuge (scale 1-5),
  max refuge size (scale 1-5), mean and SD of both of those metrics,
  depth, current (measured with flagging tape at time of deployment),
  complexity from SeaGIS analysis.
\item
  Same questions but with overall fish community, not just rockfish
\end{enumerate}

Let's make sure we have all the things we need in the metadata for the
MaxN data including adding in the refuge measurements and depth and
current.
\includegraphics{modelling_setup_files/figure-latex/unnamed-chunk-1-1.pdf}
\includegraphics{modelling_setup_files/figure-latex/unnamed-chunk-1-2.pdf}
\includegraphics{modelling_setup_files/figure-latex/unnamed-chunk-1-3.pdf}
\includegraphics{modelling_setup_files/figure-latex/unnamed-chunk-1-4.pdf}
\includegraphics{modelling_setup_files/figure-latex/unnamed-chunk-1-5.pdf}

\hypertarget{a-note-on-sponge-gardens}{%
\subsection{A note on sponge gardens}\label{a-note-on-sponge-gardens}}

We only had 3 potential gardens. One we removed because it was a boot
sponge wall and the other complication was that the sponges at hutt pair
were sliced off before the last sampling period.So I need to change the
\% cover info for that site for the last sample.

\hypertarget{plots-of-habitat-variables}{%
\subsection{Plots of habitat
variables}\label{plots-of-habitat-variables}}

\includegraphics{modelling_setup_files/figure-latex/unnamed-chunk-5-1.pdf}
\includegraphics{modelling_setup_files/figure-latex/unnamed-chunk-5-2.pdf}
\includegraphics{modelling_setup_files/figure-latex/unnamed-chunk-5-3.pdf}
\includegraphics{modelling_setup_files/figure-latex/unnamed-chunk-5-4.pdf}
\includegraphics{modelling_setup_files/figure-latex/unnamed-chunk-5-5.pdf}
\includegraphics{modelling_setup_files/figure-latex/unnamed-chunk-5-6.pdf}
\includegraphics{modelling_setup_files/figure-latex/unnamed-chunk-5-7.pdf}
\includegraphics{modelling_setup_files/figure-latex/unnamed-chunk-5-8.pdf}

\hypertarget{non-schooling-rockfish-models-no-yellowtail}{%
\subsection{Non-schooling rockfish models (no
yellowtail)}\label{non-schooling-rockfish-models-no-yellowtail}}

Question 1. Does rockfish abundance differ between rocky reefs, sponge
gardens and sponge reefs? (*Do we have enough data on sponge reefs?
Should we add them to bioherms? rocky reefs? Hmmm question for
discussion)

Could we include richness as a factor if we cant use it as a predictor
variable?

\begin{verbatim}
## `summarise()` has grouped output by 'OpCode'. You can override using the
## `.groups` argument.
\end{verbatim}

\begin{verbatim}
## # A tibble: 37 x 33
## # Groups:   OpCode [37]
##    OpCode     richn~1 Genus   Code Depth Date  habit~2  MaxN site  uniqu~3   lat
##    <fct>        <int> <fct>  <dbl> <dbl> <fct> <fct>   <dbl> <fct>   <dbl> <dbl>
##  1 birdislet~       1 Seba~ 166726  21.8 9/6/~ rocky ~     2 bird~       4  49.4
##  2 birdislet~       1 Seba~ 166726  26.8 11/2~ rocky ~     2 bird~       4  49.4
##  3 christieb~       2 Seba~ 166726  20   10/1~ rocky ~     2 chri~      13  49.5
##  4 cliffcove~       3 Seba~ 166713  26.8 11/1~ rocky ~     1 clif~      25  49.4
##  5 defencein~       2 Seba~ 166713  23.1 9/12~ sponge~     1 defe~       8  49.6
##  6 defencein~       1 Seba~ 166726  25.3 10/3~ sponge~     7 defe~       8  49.6
##  7 defencein~       2 Seba~ 166713  25.2 11/1~ sponge~     1 defe~       8  49.6
##  8 defenceof~       1 Seba~ 166726  28.6 10/3~ sponge~     5 defe~      23  49.6
##  9 defenceof~       1 Seba~ 166726  28.7 11/1~ sponge~     5 defe~      23  49.6
## 10 defenceso~       2 Seba~ 166713  23.4 9/12~ rocky ~     2 defe~       9  49.6
## # ... with 27 more rows, 22 more variables: lon <dbl>, temperature <dbl>,
## #   complexity1 <dbl>, complexity2 <dbl>, complexity3 <dbl>,
## #   current1.mpers <dbl>, current2.mpers <dbl>, current3.mpers <dbl>,
## #   mean.site.rug <dbl>, max.rug <dbl>, sd.site.rug <dbl>, mean.site.ref <dbl>,
## #   max.ref <dbl>, sd.site.ref <dbl>, mean.site.cov <dbl>, max.cov <dbl>,
## #   sd.site.cov <dbl>, sponge <fct>, complex <dbl>, current <dbl>, X <dbl>,
## #   Y <dbl>, and abbreviated variable names 1: richness, 2: habitat_type, ...
\end{verbatim}

\includegraphics{modelling_setup_files/figure-latex/unnamed-chunk-8-1.pdf}
\includegraphics{modelling_setup_files/figure-latex/unnamed-chunk-8-2.pdf}
\includegraphics{modelling_setup_files/figure-latex/unnamed-chunk-8-3.pdf}
\includegraphics{modelling_setup_files/figure-latex/unnamed-chunk-8-4.pdf}
\includegraphics{modelling_setup_files/figure-latex/unnamed-chunk-8-5.pdf}
\includegraphics{modelling_setup_files/figure-latex/unnamed-chunk-8-6.pdf}
\includegraphics{modelling_setup_files/figure-latex/unnamed-chunk-8-7.pdf}
\includegraphics{modelling_setup_files/figure-latex/unnamed-chunk-8-8.pdf}
\includegraphics{modelling_setup_files/figure-latex/unnamed-chunk-8-9.pdf}
\includegraphics{modelling_setup_files/figure-latex/unnamed-chunk-8-10.pdf}
\includegraphics{modelling_setup_files/figure-latex/unnamed-chunk-8-11.pdf}

\begin{verbatim}
## Warning: Removed 3 rows containing missing values (`geom_point()`).
\end{verbatim}

\includegraphics{modelling_setup_files/figure-latex/unnamed-chunk-8-12.pdf}

\begin{verbatim}
## Warning: Removed 6 rows containing missing values (`geom_point()`).
\end{verbatim}

\includegraphics{modelling_setup_files/figure-latex/unnamed-chunk-8-13.pdf}

\begin{verbatim}
## Warning: Removed 3 rows containing missing values (`geom_point()`).
\end{verbatim}

\includegraphics{modelling_setup_files/figure-latex/unnamed-chunk-8-14.pdf}

Simple comparison between a NULL model and one with habitat type

\begin{Shaded}
\begin{Highlighting}[]
\CommentTok{\# Simple comparison between a NULL model and one with habitat type}

\NormalTok{null }\OtherTok{\textless{}{-}} \FunctionTok{sdmTMB}\NormalTok{(}\AttributeTok{formula =}\NormalTok{ summaxn }\SpecialCharTok{\textasciitilde{}} \DecValTok{0} \SpecialCharTok{+}\NormalTok{ (}\DecValTok{1}\SpecialCharTok{|}\NormalTok{site), }\AttributeTok{mesh =}\NormalTok{ mesh, }\AttributeTok{spatial =} \StringTok{"off"}\NormalTok{, }\AttributeTok{family =} \FunctionTok{tweedie}\NormalTok{(}\AttributeTok{link =} \StringTok{"log"}\NormalTok{), }
            \AttributeTok{data =}\NormalTok{ allrf, }\AttributeTok{control =} \FunctionTok{sdmTMBcontrol}\NormalTok{(}\AttributeTok{newton\_loops =} \DecValTok{3}\NormalTok{))}

\NormalTok{hab1 }\OtherTok{\textless{}{-}} \FunctionTok{sdmTMB}\NormalTok{(}\AttributeTok{formula =}\NormalTok{ summaxn }\SpecialCharTok{\textasciitilde{}} \DecValTok{0} \SpecialCharTok{+}\NormalTok{ habitat\_type }\SpecialCharTok{+}\NormalTok{ (}\DecValTok{1}\SpecialCharTok{|}\NormalTok{site), }\AttributeTok{mesh =}\NormalTok{ mesh, }\AttributeTok{spatial =} \StringTok{"off"}\NormalTok{, }\AttributeTok{family =} \FunctionTok{tweedie}\NormalTok{(}\AttributeTok{link =} \StringTok{"log"}\NormalTok{), }\AttributeTok{data =}\NormalTok{ allrf, }\AttributeTok{control =} \FunctionTok{sdmTMBcontrol}\NormalTok{(}\AttributeTok{newton\_loops =} \DecValTok{3}\NormalTok{))}

\CommentTok{\#two habitat categories}
\NormalTok{hab2 }\OtherTok{\textless{}{-}} \FunctionTok{sdmTMB}\NormalTok{(}\AttributeTok{formula =}\NormalTok{ summaxn }\SpecialCharTok{\textasciitilde{}} \DecValTok{0} \SpecialCharTok{+}\NormalTok{ sponge }\SpecialCharTok{+}\NormalTok{ (}\DecValTok{1}\SpecialCharTok{|}\NormalTok{site), }\AttributeTok{mesh =}\NormalTok{ mesh, }\AttributeTok{spatial =} \StringTok{"off"}\NormalTok{, }\AttributeTok{family =} \FunctionTok{tweedie}\NormalTok{(}\AttributeTok{link =} \StringTok{"log"}\NormalTok{), }\AttributeTok{data =}\NormalTok{ allrf, }\AttributeTok{control =} \FunctionTok{sdmTMBcontrol}\NormalTok{(}\AttributeTok{newton\_loops =} \DecValTok{3}\NormalTok{))}

\FunctionTok{AICtab}\NormalTok{(null, hab1)}
\end{Highlighting}
\end{Shaded}

\begin{verbatim}
##      dAIC df
## hab1  0.0 6 
## null 24.8 3
\end{verbatim}

\begin{Shaded}
\begin{Highlighting}[]
\FunctionTok{sanity}\NormalTok{(hab1)}
\end{Highlighting}
\end{Shaded}

\begin{verbatim}
## v Non-linear minimizer suggests successful convergence
\end{verbatim}

\begin{verbatim}
## v Hessian matrix is positive definite
\end{verbatim}

\begin{verbatim}
## v No extreme or very small eigenvalues detected
\end{verbatim}

\begin{verbatim}
## v No gradients with respect to fixed effects are >= 0.001
\end{verbatim}

\begin{verbatim}
## v No fixed-effect standard errors are NA
\end{verbatim}

\begin{verbatim}
## v No standard errors look unreasonably large
\end{verbatim}

\begin{verbatim}
## v No sigma parameters are < 0.01
\end{verbatim}

\begin{verbatim}
## v No sigma parameters are > 100
\end{verbatim}

\begin{Shaded}
\begin{Highlighting}[]
\FunctionTok{summary}\NormalTok{(hab1)}
\end{Highlighting}
\end{Shaded}

\begin{verbatim}
## Model fit by ML ['sdmTMB']
## Formula: summaxn ~ 0 + habitat_type + (1 | site)
## Mesh: mesh
## Data: allrf
## Family: tweedie(link = 'log')
##  
##                           coef.est coef.se
## habitat_typerocky reef        1.36    0.19
## habitat_typesponge reef       2.27    0.31
## habitat_typesponge garden     1.16    0.51
## 
## Random intercepts:
##      Std. Dev.
## site      0.49
## 
## Dispersion parameter: 1.85
## Tweedie p: 1.12
## ML criterion at convergence: 112.078
## 
## See ?tidy.sdmTMB to extract these values as a data frame.
\end{verbatim}

\begin{Shaded}
\begin{Highlighting}[]
\FunctionTok{hist}\NormalTok{(}\FunctionTok{residuals}\NormalTok{(hab1)) }
\end{Highlighting}
\end{Shaded}

\includegraphics{modelling_setup_files/figure-latex/unnamed-chunk-9-1.pdf}

\begin{Shaded}
\begin{Highlighting}[]
\FunctionTok{hist}\NormalTok{(}\FunctionTok{residuals}\NormalTok{(hab1, }\AttributeTok{type =} \StringTok{"mle{-}mcmc"}\NormalTok{)) }\CommentTok{\# another way to calculate them using mcmc, got from Philina\textquotesingle{}s mesh building vignette}
\end{Highlighting}
\end{Shaded}

\begin{verbatim}
## 
## SAMPLING FOR MODEL 'tmb_generic' NOW (CHAIN 1).
## Chain 1: 
## Chain 1: Gradient evaluation took 0 seconds
## Chain 1: 1000 transitions using 10 leapfrog steps per transition would take 0 seconds.
## Chain 1: Adjust your expectations accordingly!
## Chain 1: 
## Chain 1: 
## Chain 1: Iteration:   1 / 500 [  0%]  (Warmup)
## Chain 1: Iteration:  50 / 500 [ 10%]  (Warmup)
## Chain 1: Iteration: 100 / 500 [ 20%]  (Warmup)
## Chain 1: Iteration: 150 / 500 [ 30%]  (Warmup)
## Chain 1: Iteration: 200 / 500 [ 40%]  (Warmup)
## Chain 1: Iteration: 250 / 500 [ 50%]  (Warmup)
## Chain 1: Iteration: 251 / 500 [ 50%]  (Sampling)
## Chain 1: Iteration: 300 / 500 [ 60%]  (Sampling)
## Chain 1: Iteration: 350 / 500 [ 70%]  (Sampling)
## Chain 1: Iteration: 400 / 500 [ 80%]  (Sampling)
## Chain 1: Iteration: 450 / 500 [ 90%]  (Sampling)
## Chain 1: Iteration: 500 / 500 [100%]  (Sampling)
## Chain 1: 
## Chain 1:  Elapsed Time: 0.039 seconds (Warm-up)
## Chain 1:                0.038 seconds (Sampling)
## Chain 1:                0.077 seconds (Total)
## Chain 1:
\end{verbatim}

\begin{verbatim}
## Warning: The largest R-hat is 1.06, indicating chains have not mixed.
## Running the chains for more iterations may help. See
## https://mc-stan.org/misc/warnings.html#r-hat
\end{verbatim}

\begin{verbatim}
## Warning: Bulk Effective Samples Size (ESS) is too low, indicating posterior means and medians may be unreliable.
## Running the chains for more iterations may help. See
## https://mc-stan.org/misc/warnings.html#bulk-ess
\end{verbatim}

\begin{verbatim}
## Warning: Tail Effective Samples Size (ESS) is too low, indicating posterior variances and tail quantiles may be unreliable.
## Running the chains for more iterations may help. See
## https://mc-stan.org/misc/warnings.html#tail-ess
\end{verbatim}

\includegraphics{modelling_setup_files/figure-latex/unnamed-chunk-9-2.pdf}

\begin{Shaded}
\begin{Highlighting}[]
\FunctionTok{qqnorm}\NormalTok{(}\FunctionTok{residuals}\NormalTok{(hab1)) }\CommentTok{\# qqnorm plot}
\FunctionTok{abline}\NormalTok{(}\AttributeTok{a =} \DecValTok{0}\NormalTok{, }\AttributeTok{b =} \DecValTok{1}\NormalTok{) }\CommentTok{\# add 1:1 line}
\end{Highlighting}
\end{Shaded}

\includegraphics{modelling_setup_files/figure-latex/unnamed-chunk-9-3.pdf}

\begin{Shaded}
\begin{Highlighting}[]
\NormalTok{visreg}\SpecialCharTok{::}\FunctionTok{visreg}\NormalTok{(hab1, }\StringTok{"habitat\_type"}\NormalTok{)}
\end{Highlighting}
\end{Shaded}

\includegraphics{modelling_setup_files/figure-latex/unnamed-chunk-9-4.pdf}

\hypertarget{discussion-about-what-to-do-with-sponge-reefs}{%
\subsection{Discussion about what to do with sponge
reefs?}\label{discussion-about-what-to-do-with-sponge-reefs}}

\begin{verbatim}
## # A tibble: 3 x 10
##   OpCode mean.~1 max.rug sd.si~2 mean.~3 max.ref sd.si~4 mean.~5 max.cov sd.si~6
##   <fct>    <dbl>   <dbl>   <dbl>   <dbl>   <dbl>   <dbl>   <dbl>   <dbl>   <dbl>
## 1 huttc~    3.67       5    2.07    3.17       5    1.72     1.5       4    1.22
## 2 huttc~    3.67       5    2.07    3.17       5    1.72     1.5       4    1.22
## 3 huttc~    2.33       5    2.07    2          5    1.55     1         1    0   
## # ... with abbreviated variable names 1: mean.site.rug, 2: sd.site.rug,
## #   3: mean.site.ref, 4: sd.site.ref, 5: mean.site.cov, 6: sd.site.cov
\end{verbatim}

\hypertarget{for-example-if-we-combine-the-sponge-gardens-with-the-rocky-reefs}{%
\subsubsection{For example, if we combine the sponge gardens with the
rocky
reefs\ldots{}}\label{for-example-if-we-combine-the-sponge-gardens-with-the-rocky-reefs}}

\includegraphics{modelling_setup_files/figure-latex/unnamed-chunk-11-1.pdf}

\hypertarget{question-2.-what-are-the-main-drivers-of-rockfish-abundance}{%
\subsubsection{Question 2. What are the main drivers of rockfish
abundance}\label{question-2.-what-are-the-main-drivers-of-rockfish-abundance}}

This question can be answered regardless of if there was a difference in
habitat type or not. The metrics we have include max \% cover of refuge
(scale 1-5), max refuge size (scale 1-5), mean and SD of both of those
metrics, depth, current (measured with flagging tape at time of
deployment), complexity from SeaGIS analysis.

Okay shoot I completely forgot that larsen bay was missing complexity
from the seaGIS measures and so I spent a while running this but I think
because the n is lower it lowers the overall likihood because there were
40 measures with complexity not 43. Tested complexity on smaller dataset
and now we are scrapping it

\hypertarget{model-with-max-cover-and-mean-site-cover-is-lowest-aic-but-it-is-not-fitting-super-well.-not-sure-why.-it-is-model-16.-model-2-is-the-one-with-just-max-cover-and-it-fits-well-but-has-a-difference-of-2.4.}{%
\subsubsection{model with max cover and mean site cover is lowest AIC
but it is not fitting super well. Not sure why. It is model 16. model 2
is the one with just max cover and it fits well but has a difference of
2.4.}\label{model-with-max-cover-and-mean-site-cover-is-lowest-aic-but-it-is-not-fitting-super-well.-not-sure-why.-it-is-model-16.-model-2-is-the-one-with-just-max-cover-and-it-fits-well-but-has-a-difference-of-2.4.}}

Also for model 16 the max cover and mean cover have opposite
relationships\ldots.argh! Which maybe makes sense if the variability is
super high? but then wouldn't the SD be important?

\begin{Shaded}
\begin{Highlighting}[]
\FunctionTok{AICtab}\NormalTok{(m, m1, m2, m3, m4, m5, m6, m7, m8, m9)}
\end{Highlighting}
\end{Shaded}

\begin{verbatim}
##    dAIC df
## m2  0.0 4 
## m9  0.8 4 
## m5  4.2 4 
## m3  7.9 4 
## m8 10.3 4 
## m6 12.0 4 
## m1 12.2 6 
## m4 15.3 4 
## m7 17.4 4 
## m  37.0 3
\end{verbatim}

\begin{Shaded}
\begin{Highlighting}[]
\FunctionTok{AICtab}\NormalTok{(m, m1, m2, m3, m4, m5, m6, m7, m8, m9, m11, m12, m14, m15,m16, m17, m18,m19, m20, m21, m22)}
\end{Highlighting}
\end{Shaded}

\begin{verbatim}
##     dAIC df
## m16  0.0 5 
## m20  1.6 6 
## m17  2.1 5 
## m2   2.4 4 
## m9   3.2 4 
## m14  3.8 5 
## m15  3.9 5 
## m21  4.0 8 
## m12  6.0 7 
## m5   6.6 4 
## m22  7.9 8 
## m3  10.3 4 
## m19 11.4 5 
## m11 12.2 7 
## m18 12.3 5 
## m8  12.7 4 
## m6  14.4 4 
## m1  14.6 6 
## m4  17.8 4 
## m7  19.8 4 
## m   39.4 3
\end{verbatim}

\begin{Shaded}
\begin{Highlighting}[]
\FunctionTok{tidy}\NormalTok{(m16, }\AttributeTok{conf.int =} \ConstantTok{TRUE}\NormalTok{)}
\end{Highlighting}
\end{Shaded}

\begin{verbatim}
## # A tibble: 2 x 5
##   term          estimate std.error conf.low conf.high
##   <chr>            <dbl>     <dbl>    <dbl>     <dbl>
## 1 max.cov          0.689    0.0750    0.542     0.836
## 2 mean.site.cov   -0.365    0.124    -0.608    -0.123
\end{verbatim}

\begin{Shaded}
\begin{Highlighting}[]
\FunctionTok{sanity}\NormalTok{(m16) }\CommentTok{\# a check from sdmTMB, not fitting well}
\end{Highlighting}
\end{Shaded}

\begin{verbatim}
## v Non-linear minimizer suggests successful convergence
\end{verbatim}

\begin{verbatim}
## v Hessian matrix is positive definite
\end{verbatim}

\begin{verbatim}
## v No extreme or very small eigenvalues detected
\end{verbatim}

\begin{verbatim}
## v No gradients with respect to fixed effects are >= 0.001
\end{verbatim}

\begin{verbatim}
## v No fixed-effect standard errors are NA
\end{verbatim}

\begin{verbatim}
## x `ln_tau_G` standard error may be large
\end{verbatim}

\begin{verbatim}
## i `ln_tau_G` is an internal parameter affecting `sigma_G`
\end{verbatim}

\begin{verbatim}
## i `sigma_G` is the random intercept standard deviation
\end{verbatim}

\begin{verbatim}
## i Try simplifying the model, adjusting the mesh, or adding priors
\end{verbatim}

\begin{verbatim}
## x `sigma_G` is smaller than 0.01
\end{verbatim}

\begin{verbatim}
## i Consider omitting this part of the model
\end{verbatim}

\begin{Shaded}
\begin{Highlighting}[]
\FunctionTok{sanity}\NormalTok{(m2) }\CommentTok{\# a check from sdmTMB for the model with just max.cov}
\end{Highlighting}
\end{Shaded}

\begin{verbatim}
## v Non-linear minimizer suggests successful convergence
\end{verbatim}

\begin{verbatim}
## v Hessian matrix is positive definite
\end{verbatim}

\begin{verbatim}
## v No extreme or very small eigenvalues detected
\end{verbatim}

\begin{verbatim}
## v No gradients with respect to fixed effects are >= 0.001
\end{verbatim}

\begin{verbatim}
## v No fixed-effect standard errors are NA
\end{verbatim}

\begin{verbatim}
## v No standard errors look unreasonably large
\end{verbatim}

\begin{verbatim}
## v No sigma parameters are < 0.01
\end{verbatim}

\begin{verbatim}
## v No sigma parameters are > 100
\end{verbatim}

\begin{Shaded}
\begin{Highlighting}[]
\FunctionTok{tidy}\NormalTok{(m16, }\AttributeTok{conf.int =} \ConstantTok{TRUE}\NormalTok{) }
\end{Highlighting}
\end{Shaded}

\begin{verbatim}
## # A tibble: 2 x 5
##   term          estimate std.error conf.low conf.high
##   <chr>            <dbl>     <dbl>    <dbl>     <dbl>
## 1 max.cov          0.689    0.0750    0.542     0.836
## 2 mean.site.cov   -0.365    0.124    -0.608    -0.123
\end{verbatim}

\begin{Shaded}
\begin{Highlighting}[]
\FunctionTok{tidy}\NormalTok{(m2, }\AttributeTok{conf.int =} \ConstantTok{TRUE}\NormalTok{) }
\end{Highlighting}
\end{Shaded}

\begin{verbatim}
## # A tibble: 1 x 5
##   term    estimate std.error conf.low conf.high
##   <chr>      <dbl>     <dbl>    <dbl>     <dbl>
## 1 max.cov    0.444    0.0375    0.370     0.517
\end{verbatim}

\begin{Shaded}
\begin{Highlighting}[]
\FunctionTok{hist}\NormalTok{(}\FunctionTok{residuals}\NormalTok{(m16)) }\CommentTok{\# look  at residuals}
\end{Highlighting}
\end{Shaded}

\includegraphics{modelling_setup_files/figure-latex/unnamed-chunk-13-1.pdf}

\begin{Shaded}
\begin{Highlighting}[]
\FunctionTok{hist}\NormalTok{(}\FunctionTok{residuals}\NormalTok{(m16, }\AttributeTok{type =} \StringTok{"mle{-}mcmc"}\NormalTok{)) }\CommentTok{\# another way to calculate them using mcmc, got from Philina\textquotesingle{}s mesh building vignette}
\end{Highlighting}
\end{Shaded}

\begin{verbatim}
## 
## SAMPLING FOR MODEL 'tmb_generic' NOW (CHAIN 1).
## Chain 1: 
## Chain 1: Gradient evaluation took 0 seconds
## Chain 1: 1000 transitions using 10 leapfrog steps per transition would take 0 seconds.
## Chain 1: Adjust your expectations accordingly!
## Chain 1: 
## Chain 1: 
## Chain 1: Iteration:   1 / 500 [  0%]  (Warmup)
## Chain 1: Iteration:  50 / 500 [ 10%]  (Warmup)
## Chain 1: Iteration: 100 / 500 [ 20%]  (Warmup)
## Chain 1: Iteration: 150 / 500 [ 30%]  (Warmup)
## Chain 1: Iteration: 200 / 500 [ 40%]  (Warmup)
## Chain 1: Iteration: 250 / 500 [ 50%]  (Warmup)
## Chain 1: Iteration: 251 / 500 [ 50%]  (Sampling)
## Chain 1: Iteration: 300 / 500 [ 60%]  (Sampling)
## Chain 1: Iteration: 350 / 500 [ 70%]  (Sampling)
## Chain 1: Iteration: 400 / 500 [ 80%]  (Sampling)
## Chain 1: Iteration: 450 / 500 [ 90%]  (Sampling)
## Chain 1: Iteration: 500 / 500 [100%]  (Sampling)
## Chain 1: 
## Chain 1:  Elapsed Time: 0.031 seconds (Warm-up)
## Chain 1:                0.036 seconds (Sampling)
## Chain 1:                0.067 seconds (Total)
## Chain 1:
\end{verbatim}

\begin{verbatim}
## Warning: The largest R-hat is 1.07, indicating chains have not mixed.
## Running the chains for more iterations may help. See
## https://mc-stan.org/misc/warnings.html#r-hat
\end{verbatim}

\includegraphics{modelling_setup_files/figure-latex/unnamed-chunk-13-2.pdf}

\begin{Shaded}
\begin{Highlighting}[]
\FunctionTok{qqnorm}\NormalTok{(}\FunctionTok{residuals}\NormalTok{(m16)) }\CommentTok{\# qqnorm plot}
\FunctionTok{abline}\NormalTok{(}\AttributeTok{a =} \DecValTok{0}\NormalTok{, }\AttributeTok{b =} \DecValTok{1}\NormalTok{) }\CommentTok{\# add 1:1 line}
\end{Highlighting}
\end{Shaded}

\includegraphics{modelling_setup_files/figure-latex/unnamed-chunk-13-3.pdf}

\begin{Shaded}
\begin{Highlighting}[]
\NormalTok{visreg}\SpecialCharTok{::}\FunctionTok{visreg}\NormalTok{(m16, }\StringTok{"max.cov"}\NormalTok{)}
\end{Highlighting}
\end{Shaded}

\includegraphics{modelling_setup_files/figure-latex/unnamed-chunk-13-4.pdf}

\begin{Shaded}
\begin{Highlighting}[]
\FunctionTok{hist}\NormalTok{(}\FunctionTok{residuals}\NormalTok{(m2)) }\CommentTok{\# look  at residuals}
\end{Highlighting}
\end{Shaded}

\includegraphics{modelling_setup_files/figure-latex/unnamed-chunk-13-5.pdf}

\begin{Shaded}
\begin{Highlighting}[]
\FunctionTok{hist}\NormalTok{(}\FunctionTok{residuals}\NormalTok{(m2, }\AttributeTok{type =} \StringTok{"mle{-}mcmc"}\NormalTok{)) }\CommentTok{\# another way to calculate them using mcmc, got from Philina\textquotesingle{}s mesh building vignette}
\end{Highlighting}
\end{Shaded}

\begin{verbatim}
## 
## SAMPLING FOR MODEL 'tmb_generic' NOW (CHAIN 1).
## Chain 1: 
## Chain 1: Gradient evaluation took 0 seconds
## Chain 1: 1000 transitions using 10 leapfrog steps per transition would take 0 seconds.
## Chain 1: Adjust your expectations accordingly!
## Chain 1: 
## Chain 1: 
## Chain 1: Iteration:   1 / 500 [  0%]  (Warmup)
## Chain 1: Iteration:  50 / 500 [ 10%]  (Warmup)
## Chain 1: Iteration: 100 / 500 [ 20%]  (Warmup)
## Chain 1: Iteration: 150 / 500 [ 30%]  (Warmup)
## Chain 1: Iteration: 200 / 500 [ 40%]  (Warmup)
## Chain 1: Iteration: 250 / 500 [ 50%]  (Warmup)
## Chain 1: Iteration: 251 / 500 [ 50%]  (Sampling)
## Chain 1: Iteration: 300 / 500 [ 60%]  (Sampling)
## Chain 1: Iteration: 350 / 500 [ 70%]  (Sampling)
## Chain 1: Iteration: 400 / 500 [ 80%]  (Sampling)
## Chain 1: Iteration: 450 / 500 [ 90%]  (Sampling)
## Chain 1: Iteration: 500 / 500 [100%]  (Sampling)
## Chain 1: 
## Chain 1:  Elapsed Time: 0.048 seconds (Warm-up)
## Chain 1:                0.04 seconds (Sampling)
## Chain 1:                0.088 seconds (Total)
## Chain 1:
\end{verbatim}

\begin{verbatim}
## Warning: Bulk Effective Samples Size (ESS) is too low, indicating posterior means and medians may be unreliable.
## Running the chains for more iterations may help. See
## https://mc-stan.org/misc/warnings.html#bulk-ess
\end{verbatim}

\begin{verbatim}
## Warning: Tail Effective Samples Size (ESS) is too low, indicating posterior variances and tail quantiles may be unreliable.
## Running the chains for more iterations may help. See
## https://mc-stan.org/misc/warnings.html#tail-ess
\end{verbatim}

\includegraphics{modelling_setup_files/figure-latex/unnamed-chunk-13-6.pdf}

\begin{Shaded}
\begin{Highlighting}[]
\FunctionTok{qqnorm}\NormalTok{(}\FunctionTok{residuals}\NormalTok{(m2)) }\CommentTok{\# qqnorm plot}
\FunctionTok{abline}\NormalTok{(}\AttributeTok{a =} \DecValTok{0}\NormalTok{, }\AttributeTok{b =} \DecValTok{1}\NormalTok{) }\CommentTok{\# add 1:1 line}
\end{Highlighting}
\end{Shaded}

\includegraphics{modelling_setup_files/figure-latex/unnamed-chunk-13-7.pdf}

\begin{Shaded}
\begin{Highlighting}[]
\NormalTok{visreg}\SpecialCharTok{::}\FunctionTok{visreg}\NormalTok{(m2, }\StringTok{"max.cov"}\NormalTok{)}
\end{Highlighting}
\end{Shaded}

\includegraphics{modelling_setup_files/figure-latex/unnamed-chunk-13-8.pdf}

\hypertarget{okay-i-vote-we-scrap-complexity-and-go-back-to-the-original-model-set-becuase-then-we-have-more-data-to-use-because-we-can-include-larsen-bay-see-below-dataset-as-test}{%
\subsubsection{okay I vote we scrap complexity and go back to the
original model set becuase then we have more data to use because we can
include larsen bay see below dataset as
test}\label{okay-i-vote-we-scrap-complexity-and-go-back-to-the-original-model-set-becuase-then-we-have-more-data-to-use-because-we-can-include-larsen-bay-see-below-dataset-as-test}}

\hypertarget{rockfish-richness}{%
\section{Rockfish richness}\label{rockfish-richness}}

\hypertarget{non-schooling-rockfish-models-no-yellowtail-1}{%
\subsection{Non-schooling rockfish models (no
yellowtail)}\label{non-schooling-rockfish-models-no-yellowtail-1}}

\hypertarget{question-1.-does-rockfish-richness-differ-between-rocky-reefs-sponge-gardens-and-sponge-reefs}{%
\subsubsection{Question 1. Does rockfish richness differ between rocky
reefs, sponge gardens and sponge
reefs?}\label{question-1.-does-rockfish-richness-differ-between-rocky-reefs-sponge-gardens-and-sponge-reefs}}

(*Do we have enough data on sponge reefs? Should we add them to
bioherms? rocky reefs? Hmmm question for discussion). Fitting a
negbinom2 because count data and zeros but it is not currently
converging\ldots.
\includegraphics{modelling_setup_files/figure-latex/unnamed-chunk-15-1.pdf}
\includegraphics{modelling_setup_files/figure-latex/unnamed-chunk-15-2.pdf}
\includegraphics{modelling_setup_files/figure-latex/unnamed-chunk-15-3.pdf}

Not working\ldots.

\begin{Shaded}
\begin{Highlighting}[]
\CommentTok{\# Simple comparison between a NULL model and one with habitat type}

\NormalTok{null.rich }\OtherTok{\textless{}{-}} \FunctionTok{sdmTMB}\NormalTok{(}\AttributeTok{formula =}\NormalTok{ richness }\SpecialCharTok{\textasciitilde{}} \DecValTok{0} \SpecialCharTok{+}\NormalTok{ (}\DecValTok{1}\SpecialCharTok{|}\NormalTok{site), }\AttributeTok{mesh =}\NormalTok{ mesh, }\AttributeTok{spatial =} \StringTok{"off"}\NormalTok{, }\AttributeTok{family =} \FunctionTok{nbinom2}\NormalTok{(}\AttributeTok{link =} \StringTok{"log"}\NormalTok{), }
            \AttributeTok{data =}\NormalTok{ rf.richness, }\AttributeTok{control =} \FunctionTok{sdmTMBcontrol}\NormalTok{(}\AttributeTok{newton\_loops =} \DecValTok{3}\NormalTok{))}

\NormalTok{rich1 }\OtherTok{\textless{}{-}} \FunctionTok{sdmTMB}\NormalTok{(}\AttributeTok{formula =}\NormalTok{ richness }\SpecialCharTok{\textasciitilde{}} \DecValTok{0} \SpecialCharTok{+}\NormalTok{ habitat\_type }\SpecialCharTok{+}\NormalTok{ (}\DecValTok{1}\SpecialCharTok{|}\NormalTok{site), }\AttributeTok{mesh =}\NormalTok{ mesh, }\AttributeTok{spatial =} \StringTok{"off"}\NormalTok{, }\AttributeTok{family =} \FunctionTok{nbinom2}\NormalTok{(}\AttributeTok{link =} \StringTok{"log"}\NormalTok{), }
                \AttributeTok{data =}\NormalTok{ rf.richness, }\AttributeTok{control =} \FunctionTok{sdmTMBcontrol}\NormalTok{(}\AttributeTok{newton\_loops =} \DecValTok{3}\NormalTok{))}
\end{Highlighting}
\end{Shaded}

\begin{verbatim}
## Warning in sqrt(diag(cov)): NaNs produced
\end{verbatim}

\begin{verbatim}
## Warning: The model may not have converged: non-positive-definite Hessian matrix.
\end{verbatim}

\begin{Shaded}
\begin{Highlighting}[]
\FunctionTok{AICtab}\NormalTok{(null.rich, rich1)}
\end{Highlighting}
\end{Shaded}

\begin{verbatim}
##           dAIC df
## rich1     0.0  5 
## null.rich 4.3  2
\end{verbatim}

\begin{Shaded}
\begin{Highlighting}[]
\FunctionTok{sanity}\NormalTok{(rich1)}
\end{Highlighting}
\end{Shaded}

\begin{verbatim}
## x Non-linear minimizer did not converge: do not trust this model!
\end{verbatim}

\begin{verbatim}
## i Try simplifying the model, adjusting the mesh, or adding priors
\end{verbatim}

\begin{verbatim}
## x Non-positive-definite Hessian matrix: model may not have converged
## i Try simplifying the model, adjusting the mesh, or adding priors
\end{verbatim}

\begin{verbatim}
## v No extreme or very small eigenvalues detected
\end{verbatim}

\begin{verbatim}
## v No gradients with respect to fixed effects are >= 0.001
\end{verbatim}

\begin{verbatim}
## Warning in sqrt(diag(object$cov.fixed)): NaNs produced
\end{verbatim}

\begin{verbatim}
## x `ln_phi` standard error is NA
\end{verbatim}

\begin{verbatim}
## i Try simplifying the model, adjusting the mesh, or adding priors
\end{verbatim}

\begin{verbatim}
## x `phi` standard error is NA
## i Try simplifying the model, adjusting the mesh, or adding priors
\end{verbatim}

\begin{verbatim}
## Warning in sqrt(diag(object$cov.fixed)): NaNs produced
\end{verbatim}

\begin{verbatim}
## Warning in sqrt(diag(object$cov.fixed)): NaNs produced

## Warning in sqrt(diag(object$cov.fixed)): NaNs produced

## Warning in sqrt(diag(object$cov.fixed)): NaNs produced
\end{verbatim}

\begin{verbatim}
## v No standard errors look unreasonably large
\end{verbatim}

\begin{verbatim}
## Warning in sqrt(diag(object$cov.fixed)): NaNs produced

## Warning in sqrt(diag(object$cov.fixed)): NaNs produced

## Warning in sqrt(diag(object$cov.fixed)): NaNs produced

## Warning in sqrt(diag(object$cov.fixed)): NaNs produced

## Warning in sqrt(diag(object$cov.fixed)): NaNs produced

## Warning in sqrt(diag(object$cov.fixed)): NaNs produced

## Warning in sqrt(diag(object$cov.fixed)): NaNs produced

## Warning in sqrt(diag(object$cov.fixed)): NaNs produced

## Warning in sqrt(diag(object$cov.fixed)): NaNs produced

## Warning in sqrt(diag(object$cov.fixed)): NaNs produced

## Warning in sqrt(diag(object$cov.fixed)): NaNs produced

## Warning in sqrt(diag(object$cov.fixed)): NaNs produced

## Warning in sqrt(diag(object$cov.fixed)): NaNs produced

## Warning in sqrt(diag(object$cov.fixed)): NaNs produced

## Warning in sqrt(diag(object$cov.fixed)): NaNs produced

## Warning in sqrt(diag(object$cov.fixed)): NaNs produced
\end{verbatim}

\begin{verbatim}
## v No sigma parameters are < 0.01
\end{verbatim}

\begin{verbatim}
## v No sigma parameters are > 100
\end{verbatim}

\begin{Shaded}
\begin{Highlighting}[]
\FunctionTok{summary}\NormalTok{(rich1)}
\end{Highlighting}
\end{Shaded}

\begin{verbatim}
## Model fit by ML ['sdmTMB']
## Formula: richness ~ 0 + habitat_type + (1 | site)
## Mesh: mesh
## Data: rf.richness
## Family: nbinom2(link = 'log')
\end{verbatim}

\begin{verbatim}
## Warning in sqrt(diag(object$cov.fixed)): NaNs produced

## Warning in sqrt(diag(object$cov.fixed)): NaNs produced

## Warning in sqrt(diag(object$cov.fixed)): NaNs produced

## Warning in sqrt(diag(object$cov.fixed)): NaNs produced

## Warning in sqrt(diag(object$cov.fixed)): NaNs produced

## Warning in sqrt(diag(object$cov.fixed)): NaNs produced

## Warning in sqrt(diag(object$cov.fixed)): NaNs produced

## Warning in sqrt(diag(object$cov.fixed)): NaNs produced

## Warning in sqrt(diag(object$cov.fixed)): NaNs produced

## Warning in sqrt(diag(object$cov.fixed)): NaNs produced

## Warning in sqrt(diag(object$cov.fixed)): NaNs produced

## Warning in sqrt(diag(object$cov.fixed)): NaNs produced

## Warning in sqrt(diag(object$cov.fixed)): NaNs produced

## Warning in sqrt(diag(object$cov.fixed)): NaNs produced

## Warning in sqrt(diag(object$cov.fixed)): NaNs produced

## Warning in sqrt(diag(object$cov.fixed)): NaNs produced

## Warning in sqrt(diag(object$cov.fixed)): NaNs produced

## Warning in sqrt(diag(object$cov.fixed)): NaNs produced

## Warning in sqrt(diag(object$cov.fixed)): NaNs produced

## Warning in sqrt(diag(object$cov.fixed)): NaNs produced

## Warning in sqrt(diag(object$cov.fixed)): NaNs produced

## Warning in sqrt(diag(object$cov.fixed)): NaNs produced

## Warning in sqrt(diag(object$cov.fixed)): NaNs produced

## Warning in sqrt(diag(object$cov.fixed)): NaNs produced

## Warning in sqrt(diag(object$cov.fixed)): NaNs produced

## Warning in sqrt(diag(object$cov.fixed)): NaNs produced

## Warning in sqrt(diag(object$cov.fixed)): NaNs produced

## Warning in sqrt(diag(object$cov.fixed)): NaNs produced

## Warning in sqrt(diag(object$cov.fixed)): NaNs produced

## Warning in sqrt(diag(object$cov.fixed)): NaNs produced

## Warning in sqrt(diag(object$cov.fixed)): NaNs produced

## Warning in sqrt(diag(object$cov.fixed)): NaNs produced

## Warning in sqrt(diag(object$cov.fixed)): NaNs produced

## Warning in sqrt(diag(object$cov.fixed)): NaNs produced

## Warning in sqrt(diag(object$cov.fixed)): NaNs produced

## Warning in sqrt(diag(object$cov.fixed)): NaNs produced
\end{verbatim}

\begin{verbatim}
##  
##                           coef.est coef.se
## habitat_typerocky reef        0.61    0.16
## habitat_typesponge reef       0.62    0.30
## habitat_typesponge garden     0.10    0.44
## 
## Random intercepts:
##      Std. Dev.
## site      0.29
## 
## Dispersion parameter: 10463761.22
## ML criterion at convergence: 67.603
## 
## See ?tidy.sdmTMB to extract these values as a data frame.
\end{verbatim}

\begin{Shaded}
\begin{Highlighting}[]
\FunctionTok{hist}\NormalTok{(}\FunctionTok{residuals}\NormalTok{(rich1)) }
\end{Highlighting}
\end{Shaded}

\includegraphics{modelling_setup_files/figure-latex/unnamed-chunk-16-1.pdf}

\begin{Shaded}
\begin{Highlighting}[]
\FunctionTok{hist}\NormalTok{(}\FunctionTok{residuals}\NormalTok{(rich1, }\AttributeTok{type =} \StringTok{"mle{-}mcmc"}\NormalTok{)) }\CommentTok{\# another way to calculate them using mcmc, got from Philina\textquotesingle{}s mesh building vignette}
\end{Highlighting}
\end{Shaded}

\begin{verbatim}
## Warning in sqrt(diag(object$cov.fixed)): NaNs produced

## Warning in sqrt(diag(object$cov.fixed)): NaNs produced
\end{verbatim}

\begin{verbatim}
## 
## SAMPLING FOR MODEL 'tmb_generic' NOW (CHAIN 1).
## Chain 1: 
## Chain 1: Gradient evaluation took 0 seconds
## Chain 1: 1000 transitions using 10 leapfrog steps per transition would take 0 seconds.
## Chain 1: Adjust your expectations accordingly!
## Chain 1: 
## Chain 1: 
## Chain 1: Iteration:   1 / 500 [  0%]  (Warmup)
## Chain 1: Iteration:  50 / 500 [ 10%]  (Warmup)
## Chain 1: Iteration: 100 / 500 [ 20%]  (Warmup)
## Chain 1: Iteration: 150 / 500 [ 30%]  (Warmup)
## Chain 1: Iteration: 200 / 500 [ 40%]  (Warmup)
## Chain 1: Iteration: 250 / 500 [ 50%]  (Warmup)
## Chain 1: Iteration: 251 / 500 [ 50%]  (Sampling)
## Chain 1: Iteration: 300 / 500 [ 60%]  (Sampling)
## Chain 1: Iteration: 350 / 500 [ 70%]  (Sampling)
## Chain 1: Iteration: 400 / 500 [ 80%]  (Sampling)
## Chain 1: Iteration: 450 / 500 [ 90%]  (Sampling)
## Chain 1: Iteration: 500 / 500 [100%]  (Sampling)
## Chain 1: 
## Chain 1:  Elapsed Time: 0.096 seconds (Warm-up)
## Chain 1:                0.08 seconds (Sampling)
## Chain 1:                0.176 seconds (Total)
## Chain 1:
\end{verbatim}

\begin{verbatim}
## Warning: Bulk Effective Samples Size (ESS) is too low, indicating posterior means and medians may be unreliable.
## Running the chains for more iterations may help. See
## https://mc-stan.org/misc/warnings.html#bulk-ess
\end{verbatim}

\includegraphics{modelling_setup_files/figure-latex/unnamed-chunk-16-2.pdf}

\begin{Shaded}
\begin{Highlighting}[]
\FunctionTok{qqnorm}\NormalTok{(}\FunctionTok{residuals}\NormalTok{(rich1)) }\CommentTok{\# qqnorm plot}
\FunctionTok{abline}\NormalTok{(}\AttributeTok{a =} \DecValTok{0}\NormalTok{, }\AttributeTok{b =} \DecValTok{1}\NormalTok{) }\CommentTok{\# add 1:1 line}
\end{Highlighting}
\end{Shaded}

\includegraphics{modelling_setup_files/figure-latex/unnamed-chunk-16-3.pdf}

\begin{Shaded}
\begin{Highlighting}[]
\NormalTok{visreg}\SpecialCharTok{::}\FunctionTok{visreg}\NormalTok{(rich1, }\StringTok{"habitat\_type"}\NormalTok{)}
\end{Highlighting}
\end{Shaded}

\includegraphics{modelling_setup_files/figure-latex/unnamed-chunk-16-4.pdf}

\hypertarget{lengths}{%
\section{Lengths}\label{lengths}}

Do we want to keep it as length? This would allow us to use individual
fish but we have no rockfish at two sites (boot sponge wall, field of
1000) Or do we convert to biomass so that we have one measure at each
site (then we could have a zero for the two sites with no fish)

\begin{verbatim}
## # A tibble: 1,754 x 35
##    Filename Frame Time ~1 Period Perio~2 lengt~3 Preci~4 RMS (~5 Range~6 Direc~7
##    <chr>    <dbl>   <dbl> <chr>    <dbl>   <dbl>   <dbl>   <dbl>   <dbl>   <dbl>
##  1 Left (2~ 20248    28.9 <NA>    NA         NA     1.34   0.472   1022.    4.95
##  2 Left (2~ 20338    28.9 <NA>    NA         NA     1.19   0.252    951.    2.62
##  3 Left (2~ 20580    29.1 <NA>    NA         NA     1.34   0.151   1018.    5.45
##  4 Left (2~ 20670    29.1 <NA>    NA         NA     1.20   0.14     944.   10.9 
##  5 Left (2~ 20873    29.2 <NA>    NA         NA     1.35   0.341   1024.    5.61
##  6 Left (2~ 20963    29.3 <NA>    NA         NA     1.22   0.32     961.    5.67
##  7 Left (2~ 26544    32.4 <NA>    NA        258.    0.92   1.10     816.   13.5 
##  8 Left (2~ 26574    32.4 <NA>    NA        219.    1.03   0.249    783.   25.0 
##  9 Left (2~ 27623    33.0 60min~   0.495    257.    3.28   0.375   1536.   26.6 
## 10 Left (2~ 27738    33.1 60min~   0.559    225.    2.73   0.414   1427.    6.89
## # ... with 1,744 more rows, 25 more variables: `Horz. Dir. (deg)` <dbl>,
## #   `Vert. Dir. (deg)` <dbl>, `Mid X (mm)` <dbl>, `Mid Y (mm)` <dbl>,
## #   `Mid Z (mm)` <dbl>, OpCode <chr>, TapeReader <chr>, Depth <dbl>,
## #   Comment...19 <lgl>, Date <chr>, Time <time>, habitat_type <chr>,
## #   complexity <lgl>, current_speed <lgl>, temperature <lgl>,
## #   left_camera <dbl>, right_camera <dbl>, Family <chr>, Genus <chr>,
## #   Species <chr>, Code <dbl>, Number <dbl>, Stage <chr>, Activity <chr>, ...
\end{verbatim}

\begin{verbatim}
## # A tibble: 22 x 1
##    site                  
##    <chr>                 
##  1 birdislet             
##  2 christiebuoy          
##  3 cliffcove             
##  4 defenceinshorebioherm 
##  5 defenceoffshorebioherm
##  6 defencesouthside      
##  7 halkettbayeast        
##  8 halkettbioherm        
##  9 halkettsign           
## 10 helipad               
## 11 huttcloudpair         
## 12 larsenbay             
## 13 NEbowyer              
## 14 Nechristie            
## 15 northpamrocks         
## 16 Nwpassage             
## 17 passagebioherm        
## 18 porteaurockpile       
## 19 porteauslide          
## 20 southpassage          
## 21 tennisboatramp        
## 22 tenniscourts
\end{verbatim}

\begin{verbatim}
## # A tibble: 25 x 1
##    site                  
##    <chr>                 
##  1 ansellpoint           
##  2 birdislet             
##  3 bootspongewall        
##  4 christiebuoy          
##  5 cliffcove             
##  6 defenceinshorebioherm 
##  7 defenceoffshorebioherm
##  8 defencesouthside      
##  9 fieldof1000           
## 10 halkettbayeast        
## 11 halkettbioherm        
## 12 halkettsign           
## 13 helipad               
## 14 huttcloudpair         
## 15 larsenbay             
## 16 NEbowyer              
## 17 Nechristie            
## 18 northpamrocks         
## 19 Nwpassage             
## 20 passagebioherm        
## 21 porteaurockpile       
## 22 porteauslide          
## 23 southpassage          
## 24 tennisboatramp        
## 25 tenniscourts
\end{verbatim}

\begin{verbatim}
## # A tibble: 210 x 45
##    Filename Frame Time ~1 Period Perio~2 lengt~3 Preci~4 RMS (~5 Range~6 Direc~7
##    <chr>    <dbl>   <dbl> <chr>    <dbl>   <dbl>   <dbl>   <dbl>   <dbl>   <dbl>
##  1 Left_00~   887    35.9 30min~  12.0      292.   11.4    1.20    2698.    38.3
##  2 Left_00~  1883    36.4 30min~  12.6      367.    4.00   0.485   1747.    34.5
##  3 Left (2~ 13128    24.9 60min   10.4      204.    3.05   1.29    1153.    31.3
##  4 Left (2~ 18891    28.1 60min   13.6      312.    4.13   1.12    1170.    26.1
##  5 Left (2~ 29861    34.2 60min    1.38     288.    2.40   0.148   1156.    12.6
##  6 Left (3~  2111    36.4 60min    3.59     137.    1.48   0.471    970.    11.7
##  7 Left (6~  8798    93.1 <NA>    NA        305.    1.09   0.845    926.    14.1
##  8 Left (2~ 16792    27.0 60min    0.866    300.    9.45   3.81    1898.    34.6
##  9 Left (3~   409    35.5 60min    9.39     275.    5.19   1.01    1876.    17.0
## 10 Left (3~  4232    37.6 60min   11.5      339.    3.96   0.639   1202.    21.4
## # ... with 200 more rows, 35 more variables: `Horz. Dir. (deg)` <dbl>,
## #   `Vert. Dir. (deg)` <dbl>, `Mid X (mm)` <dbl>, `Mid Y (mm)` <dbl>,
## #   `Mid Z (mm)` <dbl>, OpCode <chr>, TapeReader <chr>, Depth <dbl>,
## #   Comment...19 <lgl>, Date <chr>, Time <time>, habitat_type <chr>,
## #   complexity <lgl>, current_speed <lgl>, temperature <lgl>,
## #   left_camera <dbl>, right_camera <dbl>, Family <chr>, Genus <chr>,
## #   Species <chr>, Code <dbl>, Number <dbl>, Stage <chr>, Activity <chr>, ...
\end{verbatim}

\includegraphics{modelling_setup_files/figure-latex/unnamed-chunk-17-1.pdf}
\includegraphics{modelling_setup_files/figure-latex/unnamed-chunk-17-2.pdf}

\begin{Shaded}
\begin{Highlighting}[]
\NormalTok{null.length }\OtherTok{\textless{}{-}} \FunctionTok{sdmTMB}\NormalTok{(}\AttributeTok{formula =}\NormalTok{ length.mm }\SpecialCharTok{\textasciitilde{}} \DecValTok{0} \SpecialCharTok{+}\NormalTok{ (}\DecValTok{1}\SpecialCharTok{|}\NormalTok{site), }\AttributeTok{mesh =}\NormalTok{ mesh, }\AttributeTok{spatial =} \StringTok{"off"}\NormalTok{, }\AttributeTok{family =} \FunctionTok{gaussian}\NormalTok{(}\AttributeTok{link =} \StringTok{"identity"}\NormalTok{), }
            \AttributeTok{data =}\NormalTok{ rf.lengths, }\AttributeTok{control =} \FunctionTok{sdmTMBcontrol}\NormalTok{(}\AttributeTok{newton\_loops =} \DecValTok{3}\NormalTok{))}

\NormalTok{length.hab }\OtherTok{\textless{}{-}} \FunctionTok{sdmTMB}\NormalTok{(}\AttributeTok{formula =}\NormalTok{ length.mm }\SpecialCharTok{\textasciitilde{}} \DecValTok{0} \SpecialCharTok{+}\NormalTok{ habitat\_type }\SpecialCharTok{+}\NormalTok{ (}\DecValTok{1}\SpecialCharTok{|}\NormalTok{site), }\AttributeTok{mesh =}\NormalTok{ mesh, }\AttributeTok{spatial =} \StringTok{"off"}\NormalTok{, }\AttributeTok{family =} \FunctionTok{gaussian}\NormalTok{(}\AttributeTok{link =} \StringTok{"identity"}\NormalTok{), }
            \AttributeTok{data =}\NormalTok{ rf.lengths, }\AttributeTok{control =} \FunctionTok{sdmTMBcontrol}\NormalTok{(}\AttributeTok{newton\_loops =} \DecValTok{3}\NormalTok{))}


\FunctionTok{AICtab}\NormalTok{(null.length, length.hab)}
\end{Highlighting}
\end{Shaded}

\begin{verbatim}
##             dAIC df
## length.hab   0   5 
## null.length 68   2
\end{verbatim}

\begin{Shaded}
\begin{Highlighting}[]
\FunctionTok{sanity}\NormalTok{(length.hab)}
\end{Highlighting}
\end{Shaded}

\begin{verbatim}
## v Non-linear minimizer suggests successful convergence
\end{verbatim}

\begin{verbatim}
## v Hessian matrix is positive definite
\end{verbatim}

\begin{verbatim}
## v No extreme or very small eigenvalues detected
\end{verbatim}

\begin{verbatim}
## v No gradients with respect to fixed effects are >= 0.001
\end{verbatim}

\begin{verbatim}
## v No fixed-effect standard errors are NA
\end{verbatim}

\begin{verbatim}
## v No standard errors look unreasonably large
\end{verbatim}

\begin{verbatim}
## v No sigma parameters are < 0.01
\end{verbatim}

\begin{verbatim}
## v No sigma parameters are > 100
\end{verbatim}

\begin{Shaded}
\begin{Highlighting}[]
\FunctionTok{summary}\NormalTok{(length.hab)}
\end{Highlighting}
\end{Shaded}

\begin{verbatim}
## Model fit by ML ['sdmTMB']
## Formula: length.mm ~ 0 + habitat_type + (1 | site)
## Mesh: mesh
## Data: rf.lengths
## Family: gaussian(link = 'identity')
##  
##                           coef.est coef.se
## habitat_typerocky reef      266.94   11.68
## habitat_typesponge garden   161.27   44.04
## habitat_typesponge reef     284.37   22.66
## 
## Random intercepts:
##      Std. Dev.
## site     43.06
## 
## Dispersion parameter: 44.39
## ML criterion at convergence: 1116.518
## 
## See ?tidy.sdmTMB to extract these values as a data frame.
\end{verbatim}

\begin{Shaded}
\begin{Highlighting}[]
\FunctionTok{hist}\NormalTok{(}\FunctionTok{residuals}\NormalTok{(length.hab)) }
\end{Highlighting}
\end{Shaded}

\includegraphics{modelling_setup_files/figure-latex/unnamed-chunk-19-1.pdf}

\begin{Shaded}
\begin{Highlighting}[]
\FunctionTok{hist}\NormalTok{(}\FunctionTok{residuals}\NormalTok{(length.hab, }\AttributeTok{type =} \StringTok{"mle{-}mcmc"}\NormalTok{)) }\CommentTok{\# another way to calculate them using mcmc, got from Philina\textquotesingle{}s mesh building vignette}
\end{Highlighting}
\end{Shaded}

\begin{verbatim}
## 
## SAMPLING FOR MODEL 'tmb_generic' NOW (CHAIN 1).
## Chain 1: 
## Chain 1: Gradient evaluation took 0 seconds
## Chain 1: 1000 transitions using 10 leapfrog steps per transition would take 0 seconds.
## Chain 1: Adjust your expectations accordingly!
## Chain 1: 
## Chain 1: 
## Chain 1: Iteration:   1 / 500 [  0%]  (Warmup)
## Chain 1: Iteration:  50 / 500 [ 10%]  (Warmup)
## Chain 1: Iteration: 100 / 500 [ 20%]  (Warmup)
## Chain 1: Iteration: 150 / 500 [ 30%]  (Warmup)
## Chain 1: Iteration: 200 / 500 [ 40%]  (Warmup)
## Chain 1: Iteration: 250 / 500 [ 50%]  (Warmup)
## Chain 1: Iteration: 251 / 500 [ 50%]  (Sampling)
## Chain 1: Iteration: 300 / 500 [ 60%]  (Sampling)
## Chain 1: Iteration: 350 / 500 [ 70%]  (Sampling)
## Chain 1: Iteration: 400 / 500 [ 80%]  (Sampling)
## Chain 1: Iteration: 450 / 500 [ 90%]  (Sampling)
## Chain 1: Iteration: 500 / 500 [100%]  (Sampling)
## Chain 1: 
## Chain 1:  Elapsed Time: 0.045 seconds (Warm-up)
## Chain 1:                0.032 seconds (Sampling)
## Chain 1:                0.077 seconds (Total)
## Chain 1:
\end{verbatim}

\begin{verbatim}
## Warning: Tail Effective Samples Size (ESS) is too low, indicating posterior variances and tail quantiles may be unreliable.
## Running the chains for more iterations may help. See
## https://mc-stan.org/misc/warnings.html#tail-ess
\end{verbatim}

\includegraphics{modelling_setup_files/figure-latex/unnamed-chunk-19-2.pdf}

\begin{Shaded}
\begin{Highlighting}[]
\FunctionTok{qqnorm}\NormalTok{(}\FunctionTok{residuals}\NormalTok{(length.hab)) }\CommentTok{\# qqnorm plot}
\FunctionTok{abline}\NormalTok{(}\AttributeTok{a =} \DecValTok{0}\NormalTok{, }\AttributeTok{b =} \DecValTok{1}\NormalTok{) }\CommentTok{\# add 1:1 line}
\end{Highlighting}
\end{Shaded}

\includegraphics{modelling_setup_files/figure-latex/unnamed-chunk-19-3.pdf}

\begin{Shaded}
\begin{Highlighting}[]
\NormalTok{visreg}\SpecialCharTok{::}\FunctionTok{visreg}\NormalTok{(length.hab, }\StringTok{"habitat\_type"}\NormalTok{)}
\end{Highlighting}
\end{Shaded}

\includegraphics{modelling_setup_files/figure-latex/unnamed-chunk-19-4.pdf}

\end{document}
